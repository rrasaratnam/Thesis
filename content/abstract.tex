% !TEX root = ../main.tex
\begin{abstract}

  % Für deutschsprachige Ausarbeitungen sollte jeweils ein Abstract in deutscher und englischer Sprache verfasst werden
  % If written in german their should be two, a german and an english abstract


  % Kurze Motivation und Angabe was in der Arbeit behandelt wird. Dies wird in der Einleitung ausführlicher behandelt.
  % Brief motivation and explanation of the topic. This will typically be repeated more extensive in the "`introduction"' chapter.
  \noindent
  Die Automobilindustrie befindet sich im Wandel, wobei sich der Fokus auf die Entwicklung softwaredefinierter Fahrzeuge verlagert. Die daraus resultierende zunehmende Komplexität verteilter Systeme erfordert neue Ansätze für die Validierung in frühen Entwicklungsphasen. Ziel dieser Arbeit ist die Entwicklung eines Frameworks zur automatisierten Architekturvalidierung im bereits bestehenden ArchitekturTool des Projekts AUTOtech.agil, das zur Spezifikation von verteilten Systemen genutzt wird. Das entwickelnde Framework umfasst eine standardisierte API zur Anbindung beliebiger Algorithmen, eine Benutzeroberfläche zum Ausführen der angebundenen Algorithmen sowie eine Komponente zur Visualisierung der Ergebnisse. Als Proof of Concept wurde ein erster Validierungsalgorithmus implementiert, der einen Traceability-Check durchführt und die Nachverfolgbarkeit zwischen Anforderungen und Garantien prüft. Die Evaluation belegt, dass das Framework alle gesetzten funktionalen und nicht-funktionalen Anforderungen erfüllt und eine erweiterbare Grundlage für die automatisierte Architekturvalidierung in der Fahrzeugtechnik bietet.

\end{abstract}


\begin{otherlanguage}{english}
  \begin{abstract}
    The automotive industry is experiencing a change, with a shift in focus toward the development of software-defined vehicles. The resulting increase in complexity of distributed systems requires new approaches to validate within early development phases. The aim of this work is to develop a framework for automated architecture validation within the existing ArchitekturTool of the AUTOtech.agil project, which is used to specify distributed systems. The developed framework includes a standardized API for integrating any algorithms, a user interface for executing the integrated algorithms, and a component to visualise the results. As a proof of concept, a validation algorithm was implemented that runs a traceability check and verifies the link between requirements and guarantees. The evaluation shows that the framework meets all (non-)functional requirements and provieds an extendable basis for automated architecture validation within automotive engineering.
  \end{abstract}
\end{otherlanguage}