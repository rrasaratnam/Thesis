% !TEX root = ../main.tex

\chapter{Fazit}
\label{ch:fazit}

% Gehe auf Ziel, Aufgabenstellung aus Einleitung ein und wiederhole zusammenfassend, wie die Aufgabenstellung erfüllt wurde und was die Ergebnisse sind.
% Adress the aim and the conceptual formulation of the assignment from the introduction directly and summarise how the formulated goals were reached (or not).


Das Ziel dieser Arbeit war es, das bestehende webbasierte ArchitekturTool um ein Framework für die automatisierte Architekturvalidierung zu erweitern. Die Aufgabenstellung umfasste vier zentrale Punkte: die Implementierung einer Schnittstelle zur Anbindung von Validierungsalgorithmen, die Entwicklung einer grafischen Benutzeroberfläche zur Durchführung dieser Algorithmen, einer grafischen Benutzeroberfläche zur Visualisierung der Ergebnisse sowie die Implementierung eines ersten Validierungsalgorithmus.

Die Umsetzung dieser Aufgabenstellung erfolgte durch die Entwicklung einer serverseitigen \gls{rest}-\gls{api}, die für die Verwaltung und  Ausführung der Algorithmen zuständig ist, sowie eines komponentenbasierten Frontends, das die geforderte \gls{gui} zur Verfügung stellt.

Das Endresultat ist ein voll funktionsfähiges und erweiterbares Framework. Die Evaluation hat die korrekte Arbeitsweise des Gesamtsystems nachgewiesen: Der in der Aufgabenstellung gefordertem, Traceability-Check-Algorithmus konnte erfolgreich über die \gls{api} angebunden und die \gls{gui} ausgeführt werden. Dessen Ergebnis wurde ebenfalls korrekt dargestellt. Somit wurden alle Punkte der Aufgabenstellung erfolgreich umgesetzt und das Ziel der Arbeit erreicht.

\section{Ausblick} %(optional)
%\section{Future Work} % (optional)

% Wie könnte es weiter gehen? Dieser Abschnitt kann auch in einem eigenen Kapitel vorhergehen.
% Short description how this work could be pursued. This can be done in a separate chapter preceding the conclusion, too.

Das entwickelte Framework bildet eine solide Grundlage, die in der Zukunft auf vielfältige Art und Weise erweitert werden kann.

Ein logischer nächster Schritt ist, wie von Beginn an beabsichtigt, die Anbindung weiterer Validierungsalgorithmen. Es könnten weiterführende Algorithmen zur quantitativen Analyse implementiert werden, die beispielsweise die Netzwerkauslastung oder die Ressourcennutzung von Komponenten bewerten.  Ebenfalls bieten sich weitere Sicherheits- und Zuverlässigkeitsanaylsen, um Architekturen frühzeitig auf potenzielle Schwachstellen zu prüfen, an.

Des Weiteren bietet die Ergebnisvisualisierung Ausbaupotenzial. Die bestehende Komponente könnte um eine zusätzliche Visualisierungsform, wie Heatmaps zur Veranschaulichung von Lastenverteilung, erweitert werden. Dies würde die Interpretierbarkeit komplexer Ergebnisse verbessern und die schnelle Identifizierung von Schwachstellen ermöglichen.

Die genannten Erweiterungsmöglichkeiten legen auch eine Anpassung der grafischen Benutzeroberfläche nahe. Durch eine eventuelle Nutzerumfrage könnten weitere Funktionen entwickelt werden, um die Benutzerfreundlichkeit auch bei steigender Komplexität auf einem hohen Niveau zu halten.