% !TEX root = ../main.tex

\chapter{Einleitung}
\label{ch:introduction}
Die Autmobilindustrie befindet sich im Wandel, der sich durch den Fokus weg vom mechanikzentrierten Fahrzeug hin zum \gls{sdv} zeigt. Während früher die mechanischen Komponenten ausschlaggebend waren für den Verkauf eines Fahrzeuges, rückt heute die Software immer mehr in den Fokus und wird zum zentralen Differenzierungsmerkmal moderner Fahrzeuge \cite{Cha21}\cite{zhao2022}.

Der Fokus auf \gls{sdv} geht mit der deutlich steigenden Komplexität von \glspl{eea} einher \cite{Cha21}\cite{Pancik2018}. Moderne Fahrzeuge enthalten oft über ein hundert \glspl{ecu}, umfangreiche Verkabelung und unmengen Zeilen von Code . Diese komplexen Architekturen müssen sorgfältig geprüft werden. Da eine späte Fehlererkennung zu erheblichen Kosten führt, sollte dies bereits in frühen Entwicklungsphasen passieren \cite{Cha21}.

Für die Modellierung und Entwicklung dieser Architekturen gibt es bereits funktionsfähige, jedoch meist kostenpflichtige Tools wie PREEvision zur Verfügung, die als Industriestandard gelten. Allerdings sind fast alle dieser Tools nicht open-source und decken nicht alle Validierungsarten ab \cite{askaripoor2022architecture}\cite{schauffele2016architectural}.

Hier setzt das \textit{AUTOtech.agil}-Projekt\footnote{https://www.autotechagil.de/} an. Ziel des Projekts ist die Entwickung einer offenen Software- und \gls{eea} für zukünftige Fahrzeuggenerationen \cite{autotechagil2024}\cite{vanKempen2023}. Neben der \gls{soa} \gls{asoa} wurde auch ein webbasiertes Tool zur Spezifikation von funktionalen, Softwarebezogenen und \gls{eea} geforscht. Dieses Tool, das \textit{ArichtekturTool},  soll eine open-source und zukunftsfähige Alternative zu den bestehenden Industriestandards sein.

Jedoch fehlt diesem Tool noch eine Funktion zur automatisierten Architekurvalidierung. Genau hier liegt der Beitrag der vorliegenden Abschlussarbeit: Das ArchitekturTool um eine automatisierte Validierungsfunktion zu erweitern, um dem Endutzer beim Entwicklungsprozess zu unterstützen.
\section{Aufgabenstellung}
\label{sect:aufgabenstellung}

Das Ziel dieser Arbeit ist das ArchitekturTool um die folgenden Funktionen zu erweitern, um eine automatisierte Architekturvalidierung zu ermöglichen. Dazu wird zunächst eine \gls{api} zur Anbindung von Validierungsalgorithmen entwickelt. Anschließend wird eine \gls{gui} umgesetzt, mit der die hinterlegten Validierungsalgorithmen ausgeführt werden können. Zur Analyse beziehungsweise Interpretation der Ergebnis wird zudem eine entsprechende \gls{gui} für die Ergebnisse dieser Algorithmen implementiert. Abschließend wird ein erster Validierungsalgorithmus in die neu entwickelte Umgebung zu hinzuzufügt.
\section{Aufbau der Arbeit}
\label{sec:aufbau}

Der Aufbau dieser Arbeit ist wie folgt:

Zunächst werden im Kapitel~\ref{ch:basics} die notwendigen Grundlagen zum verständnis der Arbeit behandelt. Dazu gehören eine kurze Einführung in das ArchitekturTool, die Erläuterung der \gls{eea} im Bereich der Automobilindustrie sowie ein Überblick über die relevanten Konzepte, wie \gls{soa} und Middleware innerhalb eines Fahrzeuges. Des Weiteren werden die Motivation und Methoden hinter der Architekturvalidierung sowie der bestehende Webentwicklungs-Stack vorgestellt.

Anschließend wird im Kapitel~\ref{ch:relatedWork} der Stand der Technik betrachtet. Hier werden die in der Industrie verbreiteten Tools und Plattformen zur Architekturmodellierung und -validierung wie PREEvision, Simulink + System Composer und Capella im Detail vorgestellt und im Hinblick auf ihren Validierungsfunktionen verglichen. Abschließend werden die verschiedenen Ansätze zur Architekturvalidierung betrachtet.

Im Kapitel~\ref{ch:KuD} folgt das Konzept und Design des im Rahmen dieser Arbeit entwickelten Frameworks. Ausgehend von einer Anforderungsanalyse werden der Entwurf der Validierungs-Engine, die Systemarchitektur sowie das Konzept der \gls{api} und der \gls{gui} erläutert.

Danach wird im Kapitel~\ref{ch:imp} die praktische Umsetzung beschrieben. Dabei wird sowohl die Backend-Implementierung der Algorithmenverwaltung und Validierungs-Engine als auch die Realisierung der Frontend-Komponenten erläutert. Außerdem wird der erste Validierungsalgorithmus als \gls{poc} vorgestellt.

In Kapitel~\ref{ch:evaluation} wird das entwickelte Framework evaluiert. Mithilfe des ersten Validierungsalgorithmus werde in einem Testszenario die wichtigsten Funktionen geprüft und die Ergebnisse bewertet.

Abschließend werden im \ref{ch:fazit}.Kapitel alle Erkenntnisse zusammengetragen und mit einem Ausblick auf mögliche Weiterentwicklungen hingewiesen.