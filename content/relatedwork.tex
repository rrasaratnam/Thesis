% !TEX root = ../main.tex

\chapter{Verwandte Arbeiten}
%\chapter{Related Work}
\label{sect:relatedWork}

% Dieses Kapitel kann auch als Abschnitt im vorherigem Grundlagen Kapitel enthalten sein.
% Dies kann je nach Art der verwandten Arbeit und wie diese mit der aktuellen Arbeit verknüpft ist eine logischere Gliederung darstellen.
% This chapter can alternatively be a section of the background chapter. Depending on the relations between the actual and the related work,
% it can be a more intuitive, logic and consistent structuring to mention the related work in the previous chapter.

% Kurz vorstellen was es noch in dem Gebiet gibt und worin sich diese Arbeit davon unterscheidet
% Short presentation/summary of what has been done already in the research area and how it differs from the current thesis.

Their are many books on other guidelines describing how a bachelor or master thesis should be structured which cannot all be mentioned here.
Hence, this chapter limits its scope to a single example.
For instance the university of columbia provides some guidelines on writing theses \cite{columbia}.
According to the guideline a thesis starts with a title page with information about the title, author, department, delivery date, research mentor(s), advisors, their institutions and email adresses.
The next structuring elements are the table of contents, the list of figures and the list of tables before the introduction of the thesis.

...

The main difference between the described guideline and this document is that the guideline addresses some more questions of methodological nature to be answered.
This document however provides a more specific structure and layout template for theses written with latex at i11 in the field of computer science.
Additionally, this document gives advice on some more practical topics.
